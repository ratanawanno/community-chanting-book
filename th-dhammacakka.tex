\documentclass[
  babelLanguage=thai,
  final,
  %showtrims,
  %showwirebinding,
  %webversion,
]{chantingbook}

\usepackage{local}
\usepackage{local-thai}

\title{Dhammacakkappavattana Sutta}

\begin{document}

\mainmatter

\parittasPartSettings

\parittasChapterSettings

\chapter{Setting in Motion the Wheel of Dhamma}

\thaiText
\renewcommand{\thaiTitle}{Setting in Motion the Wheel of Dhamma}

\begin{leader}
\soloinstr{Solo introduction}

This is the first teaching of the Tathāgata on attaining to unexcelled,
perfect enlightenment.

Here is the perfect turning of the incomparable wheel of Truth,
inestimable wherever it is expounded in the world.

Disclosed here are the two extremes, and the Middle Way, with the Four Noble
Truths and the purified knowledge and vision pointed out by the Lord of
Dhamma.

Let us chant together this Sutta proclaiming the supreme, independent
enlightenment that is widely renowned as ‘The~Turning of the Wheel of
the Dhamma.’

\end{leader}

Thus have I heard.

Once when the Blessed One was staying in the deer sanctuary at
Isipatana, near Benares, he spoke to the group of five bhikkhus:

‘These two extremes, bhikkhus, should not be followed by one who has
gone forth: sensual indulgence, which is low, coarse, vulgar, ignoble,
and unprofitable; and self-torture, which is painful, ignoble, and
unprofitable.

‘Bhikkhus, by avoiding these two extremes, the Tathāgata has realized
the Middle Way, which gives vision and understanding, which leads to
calm, penetration, enlightenment, to Nibbāna.

‘And what, bhikkhus, is the Middle Way realized by the Tathāgata, which
gives vision and understanding, which leads to calm, penetration,
enlightenment, to Nibbāna?

\chapterTocSubIndentTrue
\chapter{Dhammacakkappavattana Sutta}% {{{1

\paliText
\renewcommand{\paliTitle}{Dhammacakkappavattana Sutta}

\begin{leader}
\soloinstr{Solo introduction}

\begin{solotwochants}
Anuttaraṃ abhisambodhiṃ & sambujjhitvā tathāgato\\
Pathamaṃ yaṃ adesesi & dhammacakkaṃ anuttaraṃ\\
Sammadeva pavattento & loke appativattiyaṃ\\
Yatthākkhātā ubho antā & paṭipatti ca majjhimā\\
Catūsvāriyasaccesu & visuddhaṃ ñāṇadassanaṃ\\
Desitaṃ dhammarājena & sammāsambodhikittanaṃ\\
Nāmena vissutaṃ suttaṃ & dhammacakkappavattanaṃ\\
Veyyākaraṇapāthena & saṅgītantam bhaṇāma se\\
\end{solotwochants}
\end{leader}

[Evaṃ me sutaṃ]

Ekaṃ samayaṃ bhagavā bārāṇasiyaṃ viharati isipatane migadāye. Tatra kho
bhagavā pañcavaggiye bhikkhū āmantesi:

Dve'me, bhikkhave, antā pabbajitena na sevitabbā: yo cāyaṃ kāmesu
kāma-sukh'allikānuyogo; hīno, gammo, pothujjaniko, anariyo,
anattha-sañhito; yo cāyaṃ atta-kilamathānuyogo; dukkho, anariyo,
anattha-sañhito.

Ete te, bhikkhave, ubho ante anupagamma majjhimā paṭipadā tathāgatena
abhisambuddhā cakkhukaraṇī, ñāṇakaraṇī, upasamāya, abhiññāya,
sambodhāya, nibbānāya saṃvattati.

Katamā ca sā, bhikkhave, majjhimā paṭipadā tathāgatena abhisambuddhā
cakkhukaraṇī ñāṇakaraṇī, upasamāya, abhiññāya, sambodhāya, nibbānāya
saṃvattati.

\clearpage

\thaiText
\markboth{\thaiTitle}{\rightmark}

‘It is just this Noble Eightfold Path, namely:

‘Right View, Right Intention, Right Speech, Right Action, Right
Livelihood, Right Effort, Right Mindfulness, and Right Concentration.

‘Truly, bhikkhus, this Middle Way understood by the Tathāgata produces
vision, produces knowledge, and leads to calm, penetration,
enlightenment, to Nibbāna.

‘This, bhikkhus, is the Noble Truth of dukkha:

‘Birth is dukkha, ageing is dukkha, death is dukkha, grief,
lamentation, pain, sorrow and despair are dukkha, association with the
disliked is dukkha, separation from the liked is dukkha, not to get what
one wants is dukkha. In brief, clinging to the five khandhas is dukkha.

‘This, bhikkhus, is the Noble Truth of the cause of dukkha:

‘The craving which causes rebirth and is bound up with pleasure and
lust, ever seeking fresh delight, now here, now there; namely, craving
for sense pleasure, craving for existence, and craving for annihilation.

‘This, bhikkhus, is the Noble Truth of the cessation of dukkha:

‘The complete cessation, giving up, abandonment of that craving,
complete release from that craving, and complete detachment from it.

‘This, bhikkhus, is the Noble Truth of the way leading to the cessation
of dukkha:

‘Only this Noble Eightfold Path; namely, Right View, Right Intention,
Right Speech, Right Action, Right Livelihood, Right Effort, Right
Mindfulness, and Right Concentration.

‘With the thought, “This is the Noble Truth of dukkha,” there arose in
me, bhikkhus, vision, knowledge, insight, wisdom, light, concerning
things unknown before.

\clearpage

\paliText
\markboth{\paliTitle}{\rightmark}

Ayam-eva ariyo aṭṭhaṅgiko maggo seyyathīdaṃ:

Sammā-diṭṭhi, sammā-saṅkappo, sammā-vācā, sammā-kammanto, sammā-ājīvo,
sammā-vāyāmo, sammā-sati, sammā-samādhi.

Ayaṃ kho sā, bhikkhave, majjhimā paṭipadā tathāgatena abhisambuddhā
cakkhukaraṇī ñāṇakaraṇī, upasamāya, abhiññāya, sambodhāya, nibbānāya
saṃvattati.

Idaṃ kho pana, bhikkhave, dukkhaṃ ariya-saccaṃ:

Jātipi dukkhā, jarāpi dukkhā, maranampi dukkhaṃ,
soka-parideva-dukkha-domanass'upāyāsāpi dukkhā, appiyehi sampayogo
dukkho, piyehi vippayogo dukkho, yamp'icchaṃ na labhati tampi dukkhaṃ,
saṅkhittena pañcupādānakkhandā dukkhā.

Idaṃ kho pana, bhikkhave, dukkha-samudayo ariya-saccaṃ:

Yā'yaṃ taṇhā ponobbhavikā nandi-rāga-sahagatā tatra-tatrābhinandinī
seyyathīdaṃ: kāma-taṇhā, bhava-taṇhā, vibhava-taṇhā.

Idaṃ kho pana, bhikkhave, dukkha-nirodho ariya-saccaṃ:

Yo tassā yeva taṇhāya asesa-virāga-nirodho, cāgo, paṭinissaggo, mutti,
anālayo.

Idaṃ kho pana, bhikkhave, dukkha-nirodha-gāminī paṭipadā ariya-saccaṃ:

Ayam-eva ariyo aṭṭhaṅgiko maggo seyyathīdam: sammā-diṭṭhi,
sammā-saṅkappo, sammā-vācā, sammā-kammanto, sammā-ājīvo, sammā-vāyāmo,
sammā-sati, sammā-samādhi.

\enlargethispage{\baselineskip}

[Idaṃ dukkhaṃ] ariya-saccan'ti me bhikkhave, pubbe ananussutesu dhammesu
cakkhuṃ udapādi, ñāṇaṃ udapādi, paññā udapādi, vijjā udapādi, āloko
udapādi.

\clearpage

\thaiText
\markboth{\thaiTitle}{\rightmark}

‘With the thought, “This is the Noble Truth of dukkha, and this dukkha
has to be understood,” there arose in me, bhikkhus, vision, knowledge,
insight, wisdom, light, concerning things unknown before.

‘With the thought, “This is the Noble Truth of dukkha, and this dukkha
has been understood,” there arose in me, bhikkhus, vision, knowledge,
insight, wisdom, light, concerning things unknown before.

‘With the thought, “This is the Noble Truth of the cause of dukkha,”
there arose in me, bhikkhus, vision, knowledge, insight, wisdom, light,
concerning things unknown before.

‘With the thought, “This is the Noble Truth of the cause of dukkha, and
this cause of dukkha has to be abandoned,” there arose in me, bhikkhus,
vision, knowledge, insight, wisdom, light, concerning things unknown
before.

‘With the thought, “This is the Noble Truth of the cause of dukkha, and
this cause of dukkha has been abandoned,” there arose in me, bhikkhus,
vision, knowledge, insight, wisdom, light, concerning things unknown
before.

‘With the thought, “This is the Noble Truth of the cessation of dukkha,”
there arose in me, bhikkhus, vision, knowledge, insight, wisdom, light,
concerning things unknown before.

‘With the thought, “This is the Noble Truth of the cessation of dukkha,
and this cessation of dukkha has to be realized,” there arose in me,
bhikkhus, vision, knowledge, insight, wisdom, light, concerning things
unknown before.

‘With the thought, “This is the Noble Truth of the cessation of dukkha,
and this cessation of dukkha has been realized,” there arose in me,
bhikkhus, vision, knowledge, insight, wisdom, light, concerning things
unknown before.

\clearpage

\paliText
\markboth{\paliTitle}{\rightmark}

Taṃ kho pan'idaṃ dukkhaṃ ariya-saccaṃ pariññeyyan'ti me bhikkhave, pubbe
ananussutesu dhammesu cakkhuṃ udapādi, ñāṇaṃ udapādi, paññā udapādi,
vijjā udapādi, āloko udapādi.

Taṃ kho pan'idaṃ dukkhaṃ ariya-saccaṃ pariññātan'ti me bhikkhave, pubbe
ananussutesu dhammesu cakkhuṃ udapādi, ñāṇaṃ udapādi, paññā udapādi,
vijjā udapādi, āloko udapādi.

Idaṃ dukkha-samudayo ariya-saccan'ti me bhikkhave, pubbe ananussutesu
dhammesu cakkhuṃ udapādi, ñāṇaṃ udapādi, paññā udapādi, vijjā udapādi,
āloko udapādi.

Taṃ kho pan'idaṃ dukkhasamudayo ariyasaccaṃ pahātabban'ti me bhikkhave,
pubbe ananussutesu dhammesu cakkhuṃ udapādi, ñāṇaṃ udapādi, paññā
udapādi, vijjā udapādi, āloko udapādi.

Taṃ kho pan'idaṃ dukkha-samudayo ariya-saccaṃ pahīnan'ti me bhikkhave, pubbe
ananussutesu dhammesu cakkhuṃ udapādi, ñāṇaṃ udapādi, paññā udapādi,
vijjā udapādi, āloko udapādi.

Idaṃ dukkha-nirodho ariya-saccan'ti me bhikkhave, pubbe ananussutesu
dhammesu cakkhuṃ udapādi, ñāṇaṃ udapādi, paññā udapādi, vijjā udapādi,
āloko udapādi.

Taṃ kho pan'idaṃ dukkha-nirodho ariya-saccaṃ sacchikātabban'ti me bhikkhave,
pubbe ananussutesu dhammesu cakkhuṃ udapādi, ñāṇaṃ udapādi, paññā
udapādi, vijjā, udapādi āloko udapādi.

Taṃ kho pan'idaṃ dukkha-nirodho ariya-saccaṃ sacchikatan'ti me bhikkhave,
pubbe ananussutesu dhammesu cakkhuṃ udapādi, ñāṇaṃ udapādi, paññā
udapādi, vijjā udapādi, āloko udapādi.

\clearpage

\thaiText
\markboth{\thaiTitle}{\rightmark}

‘With the thought, “This is the Noble Truth of the way leading to the
cessation of dukkha,” there arose in me, bhikkhus, vision, knowledge,
insight, wisdom, light, concerning things unknown before.

‘With the thought, “This Noble Truth of the way leading to the cessation
of dukkha has to be developed,” there arose in me, bhikkhus, vision,
knowledge, insight, wisdom, light, concerning things unknown before.

‘With the thought, “This Noble Truth of the way leading to the cessation
of dukkha has been developed,” there arose in me, bhikkhus, vision,
knowledge, insight, wisdom, light, concerning things unknown before.

‘So long, bhikkhus, as my knowledge and vision of reality regarding
these Four Noble Truths, in their three phases and twelve aspects, was
not fully clear to me, I did not declare to the world of spirits,
demons, and gods, with its seekers and sages, celestial and human
beings, the realization of incomparable, perfect enlightenment.

‘But when, bhikkhus, my knowledge and vision of reality regarding these
Four Noble Truths, in their three phases and twelve aspects, was fully
clear to me, I declared to the world of spirits, demons, and gods, with
its seekers and sages, celestial and human beings, that I had realized
incomparable, perfect enlightenment.

‘Knowledge and vision arose: “Unshakeable is my deliverance; this is
the last birth, there will be no more renewal of being.”\thinspace ’

Thus spoke the Blessed One. Glad at heart, the group of five bhikkhus
approved of the words of the Blessed One.

\clearpage

\paliText
\markboth{\paliTitle}{\rightmark}

Idaṃ dukkha-nirodha-gāminī paṭipadā ariya-saccan'ti me bhikkhave, pubbe
ananussutesu dhammesu cakkhuṃ udapādi, ñāṇaṃ udapādi, paññā udapādi,
vijjā udapādi, āloko udapādi.

Taṃ kho pan'idaṃ dukkha-nirodha-gāminī paṭipadā ariya-saccaṃ bhāvetabban'ti
me bhikkhave, pubbe ananussutesu dhammesu cakkhuṃ udapādi, ñāṇaṃ
udapādi, paññā udapādi, vijjā udapādi, āloko udapādi.

Taṃ kho pan'idaṃ dukkha-nirodha-gāminī paṭipadā ariya-saccaṃ bhāvitan'ti me
bhikkhave, pubbe ananussutesu dhammesu cakkhuṃ udapādi, ñāṇaṃ udapādi,
paññā udapādi, vijjā udapādi, āloko udapādi.

[Yāva kīvañca me bhikkhave,] imesu catūsu ariya-saccesu evan-ti-parivaṭṭaṃ
dvādas'ākāraṃ yathā-bhūtaṃ ñāṇa-dassanaṃ na suvisuddhaṃ ahosi, n'eva tāv'āhaṃ
bhikkhave, sadevake loke samārake sabrahmake sassamaṇa-brāhmaṇiyā pajāya
sadeva-manussāya anuttaraṃ sammā-sambodhiṃ abhisambuddho paccaññāsiṃ.

Yato ca kho me bhikkhave, imesu catūsu ariya-saccesu evan-ti-parivaṭṭaṃ
dvādas'ākāraṃ yathā-bhūtaṃ ñāṇa-dassanaṃ suvisuddham ahosi, ath'āham
bhikkhave, sadevake loke samārake sabrahmake sassamaṇa-brāhmaṇiyā pajāya
sadeva-manussāya anuttaraṃ sammā-sambodhiṃ abhisambuddho paccaññāsiṃ.

Ñāṇañca pana me dassanaṃ udapādi, akuppā me vimutti ayam-antimā jāti,
natthi dāni punabbhavo'ti.

Idam-avoca bhagavā. Attamanā pañcavaggiyā bhikkhū bhagavato bhāsitaṃ
abhinanduṃ.

\clearpage

\thaiText
\markboth{\thaiTitle}{\rightmark}

As this exposition was proceeding, the spotless, immaculate vision of
the Dhamma appeared to the Venerable Koṇḍañña and he knew: ‘Everything
that has the nature to arise has the nature to cease.’

When the Blessed One had set in motion the Wheel of Dhamma, the
Earthbound devas proclaimed with one voice,

‘The incomparable Wheel of Dhamma has been set in motion by the Blessed
One in the deer sanctuary at Isipatana, near Benares, and no seeker,
brahmin, celestial being, demon, god, or any other being in the world
can stop it.’

Having heard what the Earthbound devas said, the devas of the Four Great
Kings proclaimed with one voice\ldots

Having heard what the devas of the Four Great Kings said, the devas of
the Thirty-three proclaimed with one voice\ldots

Having heard what the devas of the Thirty-three said, the Yāma devas
proclaimed with one voice\ldots

Having heard what the Yāma devas said, the Devas of Delight proclaimed
with one voice\ldots

Having heard what the Devas of Delight said, the Devas Who Delight in
Creating, proclaimed with one voice\ldots

Having heard what the Devas Who Delight in Creating said, the Devas Who
Delight in the Creations of Others proclaimed with one voice\ldots

Having heard what the Devas Who Delight in the Creations of Others said,
the Brahma gods proclaimed in one voice,

‘The incomparable Wheel of Dhamma has been set in motion by the Blessed
One in the deer sanctuary at Isipatana, near Benares, and no seeker,
brahmin, celestial being, demon, god, or any other being in the world
can stop it.’

\clearpage

\paliText
\markboth{\paliTitle}{\rightmark}

Imasmiñca pana veyyākaraṇasmiṃ bhaññamāne āyasmato koṇḍaññassa virajaṃ
vītamalaṃ dhammacakkhuṃ udapādi: yaṃ kiñci samudaya-dhammaṃ sabban-taṃ
nirodha-dhamman'ti.

[Pavattite ca bhagavatā] dhammacakke bhummā devā saddamanussāvesuṃ:

Etaṃ bhagavatā bārāṇasiyaṃ isipatane migadāye anuttaraṃ dhammacakkaṃ
pavattitaṃ appaṭivattiyaṃ samaṇena vā brāhmaṇena vā devena vā mārena vā
brahmunā vā kenaci vā lokasmin'ti.

Bhummānaṃ devānaṃ saddaṃ sutvā, cātummahārājikā devā
saddamanussāvesuṃ\ldots

Cātummahārājikānaṃ devānaṃ saddaṃ sutvā, tāvatiṃsā devā
saddamanussāvesuṃ\ldots

Tāvatiṃsānaṃ devānaṃ saddaṃ sutvā, yāmā devā saddamanussāvesuṃ\ldots

Yāmānaṃ devānaṃ saddaṃ sutvā, tusitā devā saddamanussāvesuṃ\ldots

Tusitānaṃ devānaṃ saddaṃ sutvā, nimmānaratī devā saddamanussāvesum\ldots

Nimmānaratīnaṃ devānaṃ saddaṃ sutvā, paranimmitavasavattī devā
saddamanussāvesuṃ\ldots

Paranimmitavasavattīnaṃ devānaṃ saddaṃ sutvā, brahmakāyikā devā
saddamanussāvesuṃ:

Etaṃ bhagavatā bārāṇasiyaṃ isipatane migadāye anuttaraṃ dhammacakkaṃ
pavattitaṃ appaṭivattiyaṃ samaṇena vā brāhmaṇena vā devena vā mārena vā
brahmunā vā kenaci vā lokasmin'ti.

\clearpage

\thaiText
\markboth{\thaiTitle}{\rightmark}

Thus in a moment, an instant, a flash, word of the Setting in Motion of
the Wheel of Dhamma went forth up to the Brahma world, and the
ten-thousandfold universal system trembled and quaked and shook, and a
boundless, sublime radiance surpassing the power of devas appeared on
earth.

Then the Blessed One made the utterance,

‘Truly, Koṇḍañña has understood, Koṇḍañña has understood!’ Thus it was
that the Venerable Koṇḍañña got the name Aññā-Koṇḍañña: ‘Koṇḍañña Who
Understands.’

Thus ends the discourse on Setting in Motion the Wheel of Dhamma.

\clearpage

\paliText
\markboth{\paliTitle}{\rightmark}

Iti'ha tena khaṇena, tena muhuttena, yāva brahmalokā saddo abbhuggacchi.
Ayañca dasa-sahassī lokadhātu saṅkampi sampakampi sampavedhi, appamāṇo ca
oḷāro obhāso loke pāturahosi atikkammeva devānaṃ devānubhāvaṃ.

Atha kho bhagavā udānaṃ udānesi:

Aññāsi vata bho koṇḍañño, aññāsi vata bho koṇḍañño ti. Iti hidaṃ āyasmato
koṇḍaññassa aññā-koṇḍañño tveva nāmaṃ ahosī ti.

Dhammacakkappavattana-suttaṃ niṭṭhitaṃ.

\end{document}

% End of th-dhammacakka.tex