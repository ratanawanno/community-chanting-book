\documentclass[
  babelLanguage=thai,
  final,
  %showtrims,
  %showwirebinding,
  %webversion,
]{chantingbook}

\usepackage{local}
\usepackage{local-thai}

\title{Dhammacakkappavattana Sutta}

\begin{document}

\mainmatter

\usePsMarksTitleOnly

\artopttrue

\renewcommand\chapterTitleFont{\thaiFont}

\chapter{ธัมมะจักกัปปะวัตตะนะสูตร}

\thaiText
\renewcommand{\thaiTitle}{}

\begin{leader}
\soloinstr{บทขัด}

อะนุตตะรัง อะภิสัมโพธิง สัมพุชฌิตวา ตะถาคะโต\\
ปะฐะมัง ยัง อะเทเสสิ ธัมมะจักกัง อะนุตตะรัง\\
สัมมะเทวะ ปะวัตเตนโต โลเก อัปปะฏิวัตติยัง\\
ยัตถากขาตา อุโภ อันตา ปะฏิปัตติ จะ มัฌิมา\\
จะตูสวาริยะสัจเจสุ วิสุทธัง ญาณะทัสสะนัง\\
เทสิตัง ธัมมะราเชนะ สัมมาสัมโพธิกิตตะนัง\\
นาเมนะ วิสสุตัง สุตตัง ธัมมะจักกัปปะ วัตตะนัง\\
เวยยากะระณะปาเฐนะ สังคีตันตัมภะณามะ เส ฯ

\end{leader}

[เอวัมเม สุตัง]

เอกัง สะมะยัง ภะคะวา พาราณะสิยัง วิหะระติ อิสิปะตะเน
มิคะทาเย ฯ ตัต๎ระ โข ภะคะวา ปัญจะวัคคิเย ภิกขู อามันเตสิ:

เท๎วเม ภิกขะเว อันตา ปัพพะชิเตนะ นะ เสวิตัพพา โย จายัง กาเมสุ
กามะสุขัลลิกานุโยโค หีโน คัมโม โปถุชชะนิโก อะนะริโย อะนัตถะ
สัญหิโต โย จายัง อัตตะกิละมะถานุโยโค ทุกโข อะนะริโย อะนัตถะสัญหิโต ฯ

เอเต เต ภิกขะเว อุโภ อันเต อะนุปะคัมมะ มัชฌิมา ปะฏิปะทา ตะถาคะเตนะ อะภิสัมพุทธา
จักขุกะระณี ญาณะกะระณี อุปะสะมายะ อะภิญญายะ สัมโพธายะ นิพพานายะ สังวัตตะติ ฯ

\enlargethispage{\baselineskip}

กะตะมา จะ สา ภิกขะเว มัชฌิมา ปะฏิปะทา ตะถาคะเตนะ อะภิสัมพุทธา จักขุกะระณี
ญาณะกะระณี อุปะสะมายะ อะภิญญายะ สัมโพธายะ นิพพานายะ สังวัตตะติ ฯ

\clearpage

\renewcommand\chapterTitleFont{\ubuntuXMedium}

\chapter{Dhammacakkappavattana Sutta}% {{{1

\paliText
\renewcommand{\paliTitle}{}
\markboth{}{\rightmark}

\begin{leader}
\soloinstr{Solo introduction}

\begin{solotwochants}
Anuttaraṃ abhisambodhiṃ & sambujjhitvā tathāgato\\
Pathamaṃ yaṃ adesesi & dhammacakkaṃ anuttaraṃ\\
Sammadeva pavattento & loke appativattiyaṃ\\
Yatthākkhātā ubho antā & paṭipatti ca majjhimā\\
Catūsvāriyasaccesu & visuddhaṃ ñāṇadassanaṃ\\
Desitaṃ dhammarājena & sammāsambodhikittanaṃ\\
Nāmena vissutaṃ suttaṃ & dhammacakkappavattanaṃ\\
Veyyākaraṇapāthena & saṅgītantam bhaṇāma se\\
\end{solotwochants}
\end{leader}

[Evaṃ me sutaṃ]

Ekaṃ samayaṃ bhagavā bārāṇasiyaṃ viharati isipatane migadāye. Tatra kho
bhagavā pañcavaggiye bhikkhū āmantesi:

Dve'me, bhikkhave, antā pabbajitena na sevitabbā: yo cāyaṃ kāmesu
kāma-sukh'allikānuyogo; hīno, gammo, pothujjaniko, anariyo,
anattha-sañhito; yo cāyaṃ atta-kilamathānuyogo; dukkho, anariyo,
anattha-sañhito.

Ete te, bhikkhave, ubho ante anupagamma majjhimā paṭipadā tathāgatena
abhisambuddhā cakkhukaraṇī, ñāṇakaraṇī, upasamāya, abhiññāya,
sambodhāya, nibbānāya saṃvattati.

Katamā ca sā, bhikkhave, majjhimā paṭipadā tathāgatena abhisambuddhā
cakkhukaraṇī ñāṇakaraṇī, upasamāya, abhiññāya, sambodhāya, nibbānāya
saṃvattati.

\clearpage

\thaiText
\markboth{\thaiTitle}{\rightmark}

อะยะเมวะ อะริโย อัฏฐังคิโก มัคโค เสยยะถีทัง:
สัมมาทิฏฐิ สัมมาสังกัปโป สัมมาวาจา สัมมากัมมันโต สัมมาอาชีโว สัมมาวายาโม สัมมาสะติ สัมมาสะมาธิ ฯ

อะยัง โข สา ภิกขะเว มัชฌิมา ปะฏิปะทา ตะถาคะเตนะ อะภิสัมพุทธา จักขุกะระณี
ญาณะกะระณี อุปะสะมายะ อะภิญญายะ สัมโพธายะ นิพพานายะ สังวัตตะติ ฯ

อิทัง โข ปะนะ ภิกขะเว ทุกขัง อะริยะสัจจัง:

ชาติปิ ทุกขาชะราปิ ทุกขา มะระณัมปิ ทุกขัง โสกะปะริเทวะทุกขะโทมะ-
นัสสุปายาสาปิ ทุกขา อัปปิเยหิ สัมปะโยโค ทุกโข ปิเยหิ วิปปะโยโค
ทุกโข ยัมปิจฉัง นะ ละภะติ ตัมปิ ทุกขัง สังขิตเตนะ ปัญจุปาทานักขันธา ทุกขา ฯ

อิทัง โข ปะนะ ภิกขะเว ทุกขะสะมุทะโย อะริยะสัจจัง:

ยายัง ตัณหา โปโนพภะวิกา นันทิราคะสะหะคะตา ตัต๎ระ ตัต๎ราภินันทินี
เสยยะถีทัง ฯ กามะตัณหา ภะวะตัณหา วิภะวะตัณหา ฯ

อิทัง โข ปะนะ ภิกขะเว ทุกขะนิโรโธ อะริยะสัจจัง: 

โย ตัสสาเยวะ ตัณหายะ อะเสสะวิราคะนิโรโธ จาโค ปะฏินิสสัคโคมุตติ อะนาละโย ฯ

อิทัง โข ปะนะ ภิกขะเว ทุกขะนิโรธะคามินี ปะฏิปะทา อะริยะสัจจัง:

อะยะเมวะ อะริโย อัฏฐังคิโก มัคโค เสยยะถีทัง: สัมมาทิฏฐิ สัมมาสังกัปโป
สัมมาวาจา สัมมากัมมันโต สัมมาอาชีโว สัมมาวายาโม สัมมาสะติ สัมมาสะมาธิ ฯ

[อิทัง ทุกขัง] อะริยะสัจจันติ เม ภิกขะเว ปุพเพ อะนะนุสสุเตสุธัมเมสุ จักขุง
อุทะปาทิ ญาณัง อุทะปาทิ ปัญญา อุทะปาทิ วิชชา อุทะปาทิ อาโลโก อุทะปาทิ ฯ

\clearpage

\paliText
\markboth{\paliTitle}{\rightmark}

Ayam-eva ariyo aṭṭhaṅgiko maggo seyyathīdaṃ:

Sammā-diṭṭhi, sammā-saṅkappo, sammā-vācā, sammā-kammanto, sammā-ājīvo,
sammā-vāyāmo, sammā-sati, sammā-samādhi.

Ayaṃ kho sā, bhikkhave, majjhimā paṭipadā tathāgatena abhisambuddhā
cakkhukaraṇī ñāṇakaraṇī, upasamāya, abhiññāya, sambodhāya, nibbānāya
saṃvattati.

Idaṃ kho pana, bhikkhave, dukkhaṃ ariya-saccaṃ:

Jātipi dukkhā, jarāpi dukkhā, maranampi dukkhaṃ,
soka-parideva-dukkha-domanass'upāyāsāpi dukkhā, appiyehi sampayogo
dukkho, piyehi vippayogo dukkho, yamp'icchaṃ na labhati tampi dukkhaṃ,
saṅkhittena pañcupādānakkhandā dukkhā.

Idaṃ kho pana, bhikkhave, dukkha-samudayo ariya-saccaṃ:

Yā'yaṃ taṇhā ponobbhavikā nandi-rāga-sahagatā tatra-tatrābhinandinī
seyyathīdaṃ: kāma-taṇhā, bhava-taṇhā, vibhava-taṇhā.

Idaṃ kho pana, bhikkhave, dukkha-nirodho ariya-saccaṃ:

Yo tassā yeva taṇhāya asesa-virāga-nirodho, cāgo, paṭinissaggo, mutti,
anālayo.

Idaṃ kho pana, bhikkhave, dukkha-nirodha-gāminī paṭipadā ariya-saccaṃ:

Ayam-eva ariyo aṭṭhaṅgiko maggo seyyathīdam: sammā-diṭṭhi,
sammā-saṅkappo, sammā-vācā, sammā-kammanto, sammā-ājīvo, sammā-vāyāmo,
sammā-sati, sammā-samādhi.

\enlargethispage{\baselineskip}

[Idaṃ dukkhaṃ] ariya-saccan'ti me bhikkhave, pubbe ananussutesu dhammesu
cakkhuṃ udapādi, ñāṇaṃ udapādi, paññā udapādi, vijjā udapādi, āloko
udapādi.

\clearpage

\thaiText
\markboth{\thaiTitle}{\rightmark}

ตัง โข ปะนิทัง ทุกขัง อะริยะสัจจัง ปะริญเญยยันติ เม ภิกขะเว ปุพเพ อะนะนุสสุเตสุ
ธัมเมสุ จักขุง อุทะปาทิ ญาณัง อุทะปาทิ ปัญญา อุทะปาทิ วิชชา อุทะปาทิ อาโลโก อุทะปาทิ ฯ

ตัง โข ปะนิทัง ทุกขัง อะริยะสัจจัง ปะริญญาตันติ เม ภิกขะเว ปุพเพ อะนะนุสสุเตสุ ธัมเมสุ จักขุง
อุทะปาทิ ญาณัง อุทะปาทิ ปัญญา อุทะปาทิ วิชชา อุทะปาทิ อาโลโก อุทะปาทิ ฯ

อิทัง ทุกขะสะมุทะโย อะริยะสัจจันติ เม ภิกขะเว ปุพเพ อะนะนุสสุเตสุ ธัมเมสุ
จักขุง อุทะปาทิ ญาณัง อุทะปาทิ ปัญญา อุทะปาทิ วิชชา อุทะปาทิ อาโลโก อุทะปาทิ ฯ

ตัง โข ปะนิทัง ทุกขะสะมุทะโย อะริยะสัจจัง ปะหาตัพพันติเม ภิกขะเว ปุพเพ อะนะนุสสุเตสุ
ธัมเมสุ จักขุง อุทะปาทิ ญาณัง อุทะปาทิ ปัญญา อุทะปาทิ วิชชา อุทะปาทิ อาโลโก อุทะปาทิ ฯ

ตัง โข ปะนิทัง ทุกขะสะมุทะโย อะริยะสัจจัง ปะหีนันติ เม ภิกขะเว ปุพเพ อะนะนุสสุเตสุ ธัมเมสุ
จักขุง อุทะปาทิ ญาณัง อุทะปาทิ ปัญญา อุทะปาทิ วิชชา อุทะปาทิ อาโลโก อุทะปาทิ ฯ

อิทัง ทุกขะนิโรโธ อะริยะสัจจันติ เม ภิกขะเว ปุพเพ อะนะนุสสุเตสุ ธัมเมสุ จักขุง
อุทะปาทิ ญาณัง อุทะปาทิ ปัญญา อุทะปาทิ วิชชา อุทะปาทิ อาโลโก อุทะปาทิ ฯ

ตัง โข ปะนิทัง ทุกขะนิโรโธ อะริยะสัจจัง สัจฉิกาตัพพันติ เม ภิกขะเว ปุพเพ อะนะนุสสุเตสุ
ธัมเมสุ จักขุง อุทะปาทิ ญาณัง อุทะปาทิ ปัญญา อุทะปาทิ วิชชา อุทะปาทิ อาโลโก อุทะปาทิ ฯ

ตัง โข ปะนิทัง ทุกขะนิโรโธ อะริยะสัจจัง สัจฉิกะตันติ เม ภิกขะเว ปุพเพ อะนะนุสสุเตสุ ธัมเมสุ
จักขุง อุทะปาทิ ญาณัง อุทะปาทิ ปัญญา อุทะปาทิ วิชชา อุทะปาทิ อาโลโก อุทะปาทิ ฯ

\clearpage

\paliText
\markboth{\paliTitle}{\rightmark}

Taṃ kho pan'idaṃ dukkhaṃ ariya-saccaṃ pariññeyyan'ti me bhikkhave, pubbe
ananussutesu dhammesu cakkhuṃ udapādi, ñāṇaṃ udapādi, paññā udapādi,
vijjā udapādi, āloko udapādi.

Taṃ kho pan'idaṃ dukkhaṃ ariya-saccaṃ pariññātan'ti me bhikkhave, pubbe
ananussutesu dhammesu cakkhuṃ udapādi, ñāṇaṃ udapādi, paññā udapādi,
vijjā udapādi, āloko udapādi.

Idaṃ dukkha-samudayo ariya-saccan'ti me bhikkhave, pubbe ananussutesu
dhammesu cakkhuṃ udapādi, ñāṇaṃ udapādi, paññā udapādi, vijjā udapādi,
āloko udapādi.

Taṃ kho pan'idaṃ dukkhasamudayo ariyasaccaṃ pahātabban'ti me bhikkhave,
pubbe ananussutesu dhammesu cakkhuṃ udapādi, ñāṇaṃ udapādi, paññā
udapādi, vijjā udapādi, āloko udapādi.

Taṃ kho pan'idaṃ dukkha-samudayo ariya-saccaṃ pahīnan'ti me bhikkhave, pubbe
ananussutesu dhammesu cakkhuṃ udapādi, ñāṇaṃ udapādi, paññā udapādi,
vijjā udapādi, āloko udapādi.

Idaṃ dukkha-nirodho ariya-saccan'ti me bhikkhave, pubbe ananussutesu
dhammesu cakkhuṃ udapādi, ñāṇaṃ udapādi, paññā udapādi, vijjā udapādi,
āloko udapādi.

Taṃ kho pan'idaṃ dukkha-nirodho ariya-saccaṃ sacchikātabban'ti me bhikkhave,
pubbe ananussutesu dhammesu cakkhuṃ udapādi, ñāṇaṃ udapādi, paññā
udapādi, vijjā, udapādi āloko udapādi.

Taṃ kho pan'idaṃ dukkha-nirodho ariya-saccaṃ sacchikatan'ti me bhikkhave,
pubbe ananussutesu dhammesu cakkhuṃ udapādi, ñāṇaṃ udapādi, paññā
udapādi, vijjā udapādi, āloko udapādi.

\clearpage

\thaiText
\markboth{\thaiTitle}{\rightmark}

อิทัง ทุกขะนิโรธะคามินี ปะฏิปะทา อะริยะสัจจันติ เม ภิกขะเว ปุพเพ อะนะนุสสุเตสุ ธัมเมสุ
จักขุง อุทะปาทิ ญาณัง อุทะปาทิ ปัญญา อุทะปาทิ วิชชา อุทะปาทิ อาโลโก อุทะปาทิ ฯ

ตัง โข ปะนิทัง ทุกขะนิโรธะคามินี ปะฏิปะทา อะริยะสัจจัง ภาเวตัพพันติ เม ภิกขะเว ปุพเพ อะนะนุสสุเตสุ
ธัมเมสุ จักขุง อุทะปาทิ ญาณัง อุทะปาทิ ปัญญา อุทะปาทิ วิชชา อุทะปาทิ อาโลโก อุทะปาทิ ฯ

ตัง โข ปะนิทัง ทุกขะนิโรธะคามินี ปะฏิปะทา อะริยะสัจจัง ภาวิตันติ เม ภิกขะเว ปุพเพ อะนะนุสสุเตสุ
ธัมเมสุ จักขุง อุทะปาทิ ญาณัง อุทะปาทิ ปัญญา อุทะปาทิ วิชชา อุทะปาทิ อาโลโก อุทะปาทิ ฯ

[ยาวะกีวัญจะ เม ภิกขะเว] อิเมสุ จะตูสุ อะริยะสัจเจสุ เอวันติปะริวัฏฏัง
ท๎วาทะสาการัง ยะถาภูตัง ญาณะทัสสะนัง นะ สุวิสุทธัง อะโหสิ 
เนวะ ตาวาหัง ภิกขะเว สะเทวะเก โลเก สะมาระเก สะพ๎รัห๎มะเก สัสสะมะณะพ๎ราห๎มะณิยา
ปะชายะ สะเทวะมะนุสสายะ อะนุตตะรัง สัมมาสัมโพธิง อะภิสัมพุทโธ ปัจจัญญาสิง ฯ

ยะโต จะ โข เม ภิกขะเว อิเมสุ จะตูสุ อะริยะสัจเจสุ เอวันติปะริวัฏฏัง
ท๎วาทะสาการัง ยะถาภูตัง ญาณะทัสสะนัง สุวิสุทธัง อะโหสิ
อะถาหัง ภิกขะเว สะเทวะเก โลเก สะมาระเก สะพ๎รัห๎มะเก สัสสะมะณะพ๎ราห๎มะณิยา
ปะชายา สะเทวะมะนุสสายะ อะนุตตะรังสัมมาสัมโพธิง อะภิสัมพุทโธ ปัจจัญญาสิง ฯ

ญาณัญจะ ปะนะ เม ทัสสะนัง อุทะปาทิ อะกุปปา เมวิมุตติ อะยะมันติมา ชาติ นัตถิทานิ ปุนัพภะโวติ ฯ

อิทะมะโวจะ ภะคะวา ฯ อัตตะมะนา ปัญจะวัคคิยา ภิกขู ภะคะวะโต ภาสิตัง อะภินันทุง ฯ

\clearpage

\paliText
\markboth{\paliTitle}{\rightmark}

Idaṃ dukkha-nirodha-gāminī paṭipadā ariya-saccan'ti me bhikkhave, pubbe
ananussutesu dhammesu cakkhuṃ udapādi, ñāṇaṃ udapādi, paññā udapādi,
vijjā udapādi, āloko udapādi.

Taṃ kho pan'idaṃ dukkha-nirodha-gāminī paṭipadā ariya-saccaṃ bhāvetabban'ti
me bhikkhave, pubbe ananussutesu dhammesu cakkhuṃ udapādi, ñāṇaṃ
udapādi, paññā udapādi, vijjā udapādi, āloko udapādi.

Taṃ kho pan'idaṃ dukkha-nirodha-gāminī paṭipadā ariya-saccaṃ bhāvitan'ti me
bhikkhave, pubbe ananussutesu dhammesu cakkhuṃ udapādi, ñāṇaṃ udapādi,
paññā udapādi, vijjā udapādi, āloko udapādi.

[Yāva kīvañca me bhikkhave,] imesu catūsu ariya-saccesu evan-ti-parivaṭṭaṃ
dvādas'ākāraṃ yathā-bhūtaṃ ñāṇa-dassanaṃ na suvisuddhaṃ ahosi, n'eva tāv'āhaṃ
bhikkhave, sadevake loke samārake sabrahmake sassamaṇa-brāhmaṇiyā pajāya
sadeva-manussāya anuttaraṃ sammā-sambodhiṃ abhisambuddho paccaññāsiṃ.

Yato ca kho me bhikkhave, imesu catūsu ariya-saccesu evan-ti-parivaṭṭaṃ
dvādas'ākāraṃ yathā-bhūtaṃ ñāṇa-dassanaṃ suvisuddham ahosi, ath'āham
bhikkhave, sadevake loke samārake sabrahmake sassamaṇa-brāhmaṇiyā pajāya
sadeva-manussāya anuttaraṃ sammā-sambodhiṃ abhisambuddho paccaññāsiṃ.

Ñāṇañca pana me dassanaṃ udapādi, akuppā me vimutti ayam-antimā jāti,
natthi dāni punabbhavo'ti.

Idam-avoca bhagavā. Attamanā pañcavaggiyā bhikkhū bhagavato bhāsitaṃ
abhinanduṃ.

\clearpage

\thaiText
\markboth{\thaiTitle}{\rightmark}

อิมัส๎มิญจะ ปะนะ เวยยากะระณัส๎มิง ภัญญะมาเน อายัส๎มะโต โกณฑัญญัสสะ วิระชัง
วีตะมะลัง ธัมมะจักขุง อุทะปาทิ ยังกิญจิ สะมุทะยะธัมมัง สัพพันตัง นิโรธะธัมมันติ ฯ

[ปะวัตติเต จะ ภะคะวะตา] ธัมมะจักเก ภุมมา เทวา สัททะมะนุสสาเวสุง:

เอตัมภะคะวะตา พาราณะสิยัง อิสิปะตะเน มิคะทาเย อะนุตตะรัง ธัมมะจักกัง ปะวัตติตัง อัปปะฏิวัตติยัง
สะมะเณนะ วา พ๎ราห๎มะเณนะ วา เทเวนะ วา มาเรนะ วา พ๎รัห๎มุนา วา เกนะจิ วา โลกัส๎มินติ ฯ

ภุมมานัง เทวานัง สัททัง สุต๎วา จาตุมมะหาราชิกา เทวา สัททะมะนุสสาเวสุง\ldots
 
จาตุมมะหาราชิกานัง เทวานัง สัททัง สุต๎วา ตาวะติงสา เทวา สัททะมะนุสสาเวสุง\ldots
 
ตาวะติงสานัง เทวานัง สัททัง สุต๎วา ยามา เทวา สัททะมะนุสสาเวสุง\ldots

ยามานัง เทวานัง สัททัง สุต๎วา ตุสิตา เทวา สัททะมะนุสสาเวสุง\ldots

ตุสิตานัง เทวานัง สัททัง สุต๎วา นิมมานะระตี เทวา สัททะมะนุสสาเวสุง\ldots

นิมมานะระตีนัง เทวานัง สัททัง สุต๎วา ปะระนิมมิตะวะสะวัตตี เทวา สัททะมะนุสสาเวสุง\ldots

ปะระนิมมิตะวะสะวัตตีนัง เทวานัง สัททัง สุต๎วา พ๎รัห๎มะกายิกา เทวา สัททะมะนุสสาเวสุง:

เอตัมภะคะวะตา พาราณะสิยัง อิสิปะตะเน มิคะทาเย
อะนุตตะรัง ธัมมะจักกัง ปะวัตติตัง อัปปะฏิวัตติยัง สะมะเณนะ วา
พ๎ราห๎มะเณนะ วา เทเวนะ วา มาเรนะ วา พ๎รัห๎มุนา วา เกนะจิ วา โลกัส๎มินติ ฯ

\clearpage

\paliText
\markboth{\paliTitle}{\rightmark}

Imasmiñca pana veyyākaraṇasmiṃ bhaññamāne āyasmato koṇḍaññassa virajaṃ
vītamalaṃ dhammacakkhuṃ udapādi: yaṃ kiñci samudaya-dhammaṃ sabban-taṃ
nirodha-dhamman'ti.

[Pavattite ca bhagavatā] dhammacakke bhummā devā saddamanussāvesuṃ:

Etaṃ bhagavatā bārāṇasiyaṃ isipatane migadāye anuttaraṃ dhammacakkaṃ
pavattitaṃ appaṭivattiyaṃ samaṇena vā brāhmaṇena vā devena vā mārena vā
brahmunā vā kenaci vā lokasmin'ti.

Bhummānaṃ devānaṃ saddaṃ sutvā, cātummahārājikā devā
saddamanussāvesuṃ\ldots

Cātummahārājikānaṃ devānaṃ saddaṃ sutvā, tāvatiṃsā devā
saddamanussāvesuṃ\ldots

Tāvatiṃsānaṃ devānaṃ saddaṃ sutvā, yāmā devā saddamanussāvesuṃ\ldots

Yāmānaṃ devānaṃ saddaṃ sutvā, tusitā devā saddamanussāvesuṃ\ldots

Tusitānaṃ devānaṃ saddaṃ sutvā, nimmānaratī devā saddamanussāvesum\ldots

Nimmānaratīnaṃ devānaṃ saddaṃ sutvā, paranimmitavasavattī devā
saddamanussāvesuṃ\ldots

Paranimmitavasavattīnaṃ devānaṃ saddaṃ sutvā, brahmakāyikā devā
saddamanussāvesuṃ:

Etaṃ bhagavatā bārāṇasiyaṃ isipatane migadāye anuttaraṃ dhammacakkaṃ
pavattitaṃ appaṭivattiyaṃ samaṇena vā brāhmaṇena vā devena vā mārena vā
brahmunā vā kenaci vā lokasmin'ti.

\clearpage

\thaiText
\markboth{\thaiTitle}{\rightmark}

อิติหะ เตนะ ขะเณนะ เตนะ มุหุตเตนะ ยาวะ พ๎รัห๎มะโลกา สัทโท อัพภุคคัจฉิ 
อะยัญจะ ทะสะสะหัสสี โลกะธาตุ สังกัมปิสัมปะกัมปิ สัมปะเวธิ 
อัปปะมาโณ จะ โอฬาโร โอภาโส โลเกปาตุระโหสิ อะติกกัมเมวะ เทวานัง เทวานุภาวัง ฯ

อะถะโข ภะคะวา อุทานัง อุทาเนสิ:

อัญญาสิ วะตะ โภ โกณฑัญโญ อัญญาสิ วะตะ โภ โกณฑัญโญติ \\
อิติหิทัง อายัส๎มะโต โกณฑัญญัสสะ อัญญาโกณฑัญโญ เต๎ววะ นามัง อะโหสีติ ฯ 

ธัมมะจักกัปปะวัตตะนะสุตตัง นิฏฐิตัง

\clearpage

\paliText
\markboth{\paliTitle}{\rightmark}

Iti'ha tena khaṇena, tena muhuttena, yāva brahmalokā saddo abbhuggacchi.
Ayañca dasa-sahassī lokadhātu saṅkampi sampakampi sampavedhi, appamāṇo ca
oḷāro obhāso loke pāturahosi atikkammeva devānaṃ devānubhāvaṃ.

Atha kho bhagavā udānaṃ udānesi:

Aññāsi vata bho koṇḍañño, aññāsi vata bho koṇḍañño ti. Iti hidaṃ āyasmato
koṇḍaññassa aññā-koṇḍañño tveva nāmaṃ ahosī ti.

Dhammacakkappavattana-suttaṃ niṭṭhitaṃ.

\end{document}

% End of th-dhammacakka.tex
